\CourseCode{ECE101}
\CourseName{Sample course}
\CourseCategory{PM}
\CourseType{TC}
\CourseLTPXHours{3}{1}{0}{0}
\CourseCredits{4}
\CoursePrerequisite{Course Code of Sample Pre-requisite Course Code of Sample Pre-requisite}

\CourseDescription{
This course introduces fundamental circuit analysis methods for linear networks.
}

\CourseObjectives{
\begin{itemize}
  \item Develop the ability to model simple electrical networks using ideal elements.
  \item Apply systematic methods to solve DC and AC linear circuits.
  \item Interpret circuit behaviour using standard engineering metrics.
  \item Develop the ability to model simple electrical networks using ideal elements.
  \item Apply systematic methods to solve DC and AC linear circuits.
  \item Interpret circuit behaviour using standard engineering metrics.
\end{itemize}
}

\CourseOutcomes{
  \item Analyse linear circuits using nodal and mesh methods.
  \item Determine steady-state AC responses using phasor techniques.
  \item Evaluate power in AC circuits using appropriate quantities.
}

\CourseNBAArticulation{
  {3, 2, 1, 1, -, -, -, -, -, -, -, 2, 1, 1},
  {2, 3, 2, 1, -, -, -,2, 1, 1, -, -, -, -},
  {1, 2, 3, 2, 1, -, -, -, -, -, 2, 1, 1,-}
}
\CourseABETArticulation{
  {3, 2, 1, 1, -, -, -, 2, 1, 1},
  {2, 3, 2, 1, -, 2, 1, 1,  -, -},
  {1, 2, 3, 2, 1, -, -,  1, 1,-}
}

\CourseSyllabus{
\begin{itemize}
  \item Circuit variables and elements; independent and dependent sources.
  \item Network theorems: superposition, Thevenin, Norton, maximum power transfer.
  \item Nodal and mesh analysis; source transformations.
  \item First-order circuits: RC/RL transient response (basic form).
  \item Sinusoidal steady state; phasors; impedance and admittance.
  \item AC power: real, reactive, apparent power; power factor.
\end{itemize}
}
\CourseUnit{
  \UnitNumber{5}
  \UnitTitle{MOS Transistors and IC Fabrication}

  \TheoryHours{9}
  \LabHours{6}
  \XHours{0}

  \UnitCOs{CO4, CO5}

  \TheoryContent{
    \begin{itemize}
      \item MOS capacitor and MOSFET operation
        \begin{itemize}
          \item MOS capacitor: accumulation, depletion, inversion
          \item Threshold voltage concept
          \item MOSFET I–V: cutoff, linear, saturation
          \item MOSFET as a switch
        \end{itemize}

      \item IC fabrication technology
        \begin{itemize}
          \item IC definition and classification
          \item IC design flow
          \item IC Fabrication: Purpose, Various stages
          \item IC Packaging and Bonding
        \end{itemize}
    \end{itemize}
  }

  \LabContent{
    \begin{itemize}
      \item MOSFET Transfer and Output Characteristics
        \begin{itemize}
          \item Measure MOSFET transfer and output characteristics to estimate threshold voltage.
          \item Design a test circuit to observe MOSFET switching behaviour.
          \item Interpret measured parameters using datasheet specifications.
        \end{itemize}

      \item IC Fabrication Flow and Device Cross-Section Sketching
        \begin{itemize}
          \item Develop a logical IC fabrication process flow.
          \item Draw device cross-sections after key fabrication steps.
          \item Use standard IC fabrication terminology and layer conventions.
        \end{itemize}
    \end{itemize}
  }
}

\CourseReferences{
\textbf{Textbooks}
\begin{enumerate}
  \item Sedra, A. S., and Smith, K. C., \textit{Microelectronic Circuits}, 7th ed., Oxford University Press, 2016.
  \item Boylestad, R. L., and Nashelsky, L., \textit{Electronic Devices and Circuit Theory}, 11th ed., Pearson, 2013.
\end{enumerate}

\textbf{References}
\begin{enumerate}
  \item Millman, J., and Halkias, C., \textit{Integrated Electronics}, McGraw-Hill, 1972.
  \item IEEE Std 315-1975, Graphic Symbols for Electrical and Electronics Diagrams.
\end{enumerate}
}