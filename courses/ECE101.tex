\CourseCode{ECE101}
\CourseName{Sample course}
\CourseCategory{PM}
\CourseType{TC}
\CourseLTPXHours{3}{0}{2}{0}
\CourseCredits{4}
\CoursePrerequisite{Course Code of Sample Pre-requisite Course Code of Sample Pre-requisite}

\CourseDescription{
This course introduces fundamental circuit analysis methods for linear networks.
}

\CourseObjectives{
\begin{itemize}
  \item Develop the ability to model simple electrical networks using ideal elements.
  \item Apply systematic methods to solve DC and AC linear circuits.
  \item Interpret circuit behaviour using standard engineering metrics.
  \item Develop the ability to model simple electrical networks using ideal elements.
  \item Apply systematic methods to solve DC and AC linear circuits.
  \item Interpret circuit behaviour using standard engineering metrics.
\end{itemize}
}

\CourseOutcomes{
  \item Analyse linear circuits using nodal and mesh methods.
  \item Determine steady-state AC responses using phasor techniques.
  \item Evaluate power in AC circuits using appropriate quantities.
}

\CourseNBAArticulation{
  {3, 2, 1, 1, -, -, -, -, -, -, -, 2, 1, 1},
  {2, 3, 2, 1, -, -, -,2, 1, 1, -, -, -, -},
  {1, 2, 3, 2, 1, -, -, -, -, -, 2, 1, 1,-}
}
\CourseABETArticulation{
  {3, 2, 1, 1, -, -, -, 2, 1, 1},
  {2, 3, 2, 1, -, 2, 1, 1,  -, -},
  {1, 2, 3, 2, 1, -, -,  1, 1,-}
}

\CourseSyllabus{
\begin{itemize}
  \item Circuit variables and elements; independent and dependent sources.
  \item Network theorems: superposition, Thevenin, Norton, maximum power transfer.
  \item Nodal and mesh analysis; source transformations.
  \item First-order circuits: RC/RL transient response (basic form).
  \item Sinusoidal steady state; phasors; impedance and admittance.
  \item AC power: real, reactive, apparent power; power factor.
\end{itemize}
}
\CourseUnit{
  \UnitNumber{1}
  \UnitTitle{Semiconductor Physics}

  \TheoryHours{9}
  \LabHours{6}
  \XHours{0}

  \UnitCOs{CO1}

  \TheoryContent{
    \begin{itemize}
      \item Energy band formation
        \begin{itemize}
          \item E-k diagrams, Energy band formation: conductors, semiconductors, insulators
        \end{itemize}

      \item Intrinsic and extrinsic semiconductors
        \begin{itemize}
          \item Intrinsic and extrinsic semiconductors, their energy bands
        \end{itemize}

      \item Doping, impurity ionization, Fermi level
        \begin{itemize}
          \item Doping, impurity ionization, Fermi level
        \end{itemize}

      \item Carrier transport; carrier generation and recombination
        \begin{itemize}
          \item Carrier transport: drift, mobility, diffusion
          \item Carrier generation and recombination
        \end{itemize}

      \item Resistivity; sheet resistance; Poisson and continuity equations
        \begin{itemize}
          \item Resistivity of doped semiconductors; sheet resistance fundamentals
          \item Poisson and continuity equations
        \end{itemize}
    \end{itemize}
  }

  \LabContent{
    \begin{itemize}
      \item Simulation of Energy Band Models and Material Resistivity Trends
        \begin{itemize}
          \item Define realistic semiconductor materials, doping concentrations, and temperature ranges using standard material parameters.
          \item Compute and plot resistivity variation with doping concentration and temperature, ensuring physically meaningful results.
          \item Illustrate qualitative energy band diagrams and analyse Fermi level shift with doping type and level.
        \end{itemize}

      \item Carrier Concentration vs Temperature Visualization
        \begin{itemize}
          \item Compute carrier concentration variation with temperature for intrinsic and doped semiconductors.
          \item Select doping concentrations to satisfy a target resistivity condition at room temperature.
          \item Plot and analyse carrier concentration trends over a practical temperature range.
        \end{itemize}
    \end{itemize}
  }
}

\CourseUnit{
  \UnitNumber{2}
  \UnitTitle{PN Junction and Diode Characteristics}

  \TheoryHours{9}
  \LabHours{6}
  \XHours{0}

  \UnitCOs{CO2}

  \TheoryContent{
    \begin{itemize}
      \item PN junction and biasing
        \begin{itemize}
          \item PN junction formation, equilibrium, depletion region
          \item Forward and reverse bias characteristics
        \end{itemize}

      \item Diode characteristics and models
        \begin{itemize}
          \item Diode I–V equation
          \item Temperature effects on diode behaviour
          \item Junction capacitances
          \item Spice modelling of diodes, diode small signal model
        \end{itemize}

      \item Breakdown mechanisms and special diodes
        \begin{itemize}
          \item Breakdown mechanisms: Zener, avalanche
          \item Zener and Schottky diode basics
        \end{itemize}

      \item Optoelectronic diodes
        \begin{itemize}
          \item Photodiode operation, responsivity, dark current
          \item LED operation
          \item Solar cell: photocurrent, open-circuit voltage
        \end{itemize}

      \item Diode datasheets
        \begin{itemize}
          \item Interpretation of diode datasheets
        \end{itemize}
    \end{itemize}
  }

  \LabContent{
    \begin{itemize}
      \item PN Junction and Zener Diode I--V Characteristics
        \begin{itemize}
          \item Design a safe experimental setup to measure forward and reverse I--V characteristics within datasheet limits.
          \item Record and plot I--V characteristics to identify key operating regions.
          \item Compare experimentally obtained parameters with datasheet specifications.
        \end{itemize}

      \item LED and Photodiode I--V Characteristics under Illumination
        \begin{itemize}
          \item Measure I--V characteristics of LEDs and photodiodes under controlled illumination.
          \item Select appropriate sensing resistance and analyse wavelength-dependent response.
          \item Interpret results using responsivity and wavelength data from datasheets.
        \end{itemize}
    \end{itemize}
  }
}

\CourseUnit{
  \UnitNumber{3}
  \UnitTitle{Diode Circuits and Applications}

  \TheoryHours{9}
  \LabHours{6}
  \XHours{0}

  \UnitCOs{CO3}

  \TheoryContent{
    \begin{itemize}
      \item Rectifiers and ripple
        \begin{itemize}
          \item Half-wave, full-wave, and bridge rectifiers
          \item Ripple factor
        \end{itemize}

      \item Filter circuits
        \begin{itemize}
          \item Filter circuits (capacitor filter, RC basics)
        \end{itemize}

      \item Wave shaping circuits
        \begin{itemize}
          \item Clippers and clampers
          \item Voltage multiplier
        \end{itemize}

      \item Zener diode applications
        \begin{itemize}
          \item Zener diode as voltage limiter/regulator
        \end{itemize}

      \item Load-line and component considerations
        \begin{itemize}
          \item Load-line analysis of Diodes
          \item Component tolerances and derating fundamentals
        \end{itemize}
    \end{itemize}
  }

  \LabContent{
    \begin{itemize}
      \item Rectifier with Filter Design to Achieve Given Ripple Specification
        \begin{itemize}
          \item Design a rectifier and filter circuit to meet a specified ripple requirement.
          \item Select component ratings based on voltage and current constraints.
          \item Measure output waveform and ripple and validate against design targets.
        \end{itemize}

      \item Design of a Zener Voltage Regulator
        \begin{itemize}
          \item Design a Zener regulator to maintain approximately constant output voltage.
          \item Select series resistance based on load and Zener current requirements.
          \item Evaluate regulation performance under varying load conditions.
        \end{itemize}

      \item Clipper and Clamper Circuit Realisation and Analysis
        \begin{itemize}
          \item Design clipper and clamper circuits to achieve specified voltage limits.
          \item Analyse circuit behaviour considering diode forward voltage.
          \item Validate output waveforms using practical component parameters.
        \end{itemize}
    \end{itemize}
  }
}

\CourseUnit{
  \UnitNumber{4}
  \UnitTitle{Bipolar Junction Transistors (BJTs)}

  \TheoryHours{9}
  \LabHours{6}
  \XHours{0}

  \UnitCOs{CO4}

  \TheoryContent{
    \begin{itemize}
      \item BJT structure and operation
        \begin{itemize}
          \item BJT structure: NPN and PNP
          \item Carrier flow and transistor action
        \end{itemize}

      \item Operating regions and characteristics
        \begin{itemize}
          \item Operating regions: cutoff, active, saturation
          \item Large-signal output characteristics
          \item Early effect
        \end{itemize}

      \item BJT parameters
        \begin{itemize}
          \item Emitter efficiency, base transport factor
        \end{itemize}

      \item Switching behaviour
        \begin{itemize}
          \item BJT switching characteristics: rise time, fall time, storage time
        \end{itemize}
    \end{itemize}
  }

  \LabContent{
    \begin{itemize}
      \item BJT CE Characteristics and Parameter Extraction
        \begin{itemize}
          \item Design an experimental setup to measure CE characteristics safely.
          \item Extract transistor parameters across different operating regions.
          \item Verify measured values against datasheet limits.
        \end{itemize}

      \item BJT as a Switch Using Pulse Input
        \begin{itemize}
          \item Design a BJT switch circuit for a specified load and supply.
          \item Select base and collector resistances to ensure saturation.
          \item Analyse switching behaviour using pulse input and output waveforms.
        \end{itemize}
    \end{itemize}
  }
}
\CourseUnit{
  \UnitNumber{5}
  \UnitTitle{MOS Transistors and IC Fabrication}

  \TheoryHours{9}
  \LabHours{6}
  \XHours{0}

  \UnitCOs{CO4, CO5}

  \TheoryContent{
    \begin{itemize}
      \item MOS capacitor and MOSFET operation
        \begin{itemize}
          \item MOS capacitor: accumulation, depletion, inversion
          \item Threshold voltage concept
          \item MOSFET I–V: cutoff, linear, saturation
          \item MOSFET as a switch
        \end{itemize}

      \item IC fabrication technology
        \begin{itemize}
          \item IC definition and classification
          \item IC design flow
          \item IC Fabrication: Purpose, Various stages
          \item IC Packaging and Bonding
        \end{itemize}
    \end{itemize}
  }

  \LabContent{
    \begin{itemize}
      \item MOSFET Transfer and Output Characteristics
        \begin{itemize}
          \item Measure MOSFET transfer and output characteristics to estimate threshold voltage.
          \item Design a test circuit to observe MOSFET switching behaviour.
          \item Interpret measured parameters using datasheet specifications.
        \end{itemize}

      \item IC Fabrication Flow and Device Cross-Section Sketching
        \begin{itemize}
          \item Develop a logical IC fabrication process flow.
          \item Draw device cross-sections after key fabrication steps.
          \item Use standard IC fabrication terminology and layer conventions.
        \end{itemize}
    \end{itemize}
  }
}

\CourseReferences{
\textbf{Textbooks}
\begin{enumerate}
  \item Sedra, A. S., and Smith, K. C., \textit{Microelectronic Circuits}, 7th ed., Oxford University Press, 2016.
  \item Boylestad, R. L., and Nashelsky, L., \textit{Electronic Devices and Circuit Theory}, 11th ed., Pearson, 2013.
\end{enumerate}

\textbf{References}
\begin{enumerate}
  \item Millman, J., and Halkias, C., \textit{Integrated Electronics}, McGraw-Hill, 1972.
  \item IEEE Std 315-1975, Graphic Symbols for Electrical and Electronics Diagrams.
\end{enumerate}
}