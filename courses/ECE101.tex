\BeginCourse{ECE101}

\CourseCode{ECE101}
\CourseName{Circuit Analysis}
\CourseCategory{PM}
\CourseType{TC}
\CourseLTPXHours{3}{1}{0}{0}
\CourseCredits{4}
\CoursePrerequisite{Course Code of Sample Pre-requisite}

\CourseDescription{
This course introduces fundamental circuit analysis methods for linear networks.
}

\CourseObjectives{
\begin{itemize}
  \item Develop the ability to model simple electrical networks using ideal elements.
  \item Apply systematic methods to solve DC and AC linear circuits.
  \item Interpret circuit behaviour using standard engineering metrics.
\end{itemize}
}

\CourseOutcomes{
  \item Analyse linear circuits using nodal and mesh methods.
  \item Determine steady-state AC responses using phasor techniques.
  \item Evaluate power in AC circuits using appropriate quantities.
}

\CourseNBAArticulation{
  {3, 2, 1, 1, -, -, -, -, -, -, -},
  {2, 3, 2, 1, -, -, -, -, -, -, -},
  {1, 2, 3, 2, 1, -, -, -, -, -, -}
}

\CourseSyllabus{
\begin{itemize}
  \item Circuit variables and elements; independent and dependent sources.
  \item Network theorems: superposition, Thevenin, Norton, maximum power transfer.
  \item Nodal and mesh analysis; source transformations.
  \item First-order circuits: RC/RL transient response (basic form).
  \item Sinusoidal steady state; phasors; impedance and admittance.
  \item AC power: real, reactive, apparent power; power factor.
\end{itemize}
}

\EndCourse