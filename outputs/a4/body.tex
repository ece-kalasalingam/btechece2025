% ==================================================
% AUTO-GENERATED FILE - DO NOT EDIT
% GENERATED-BY : scripts/process_courses.py
% VERSION      : v1.0
% VIEW         : A4
% COURSE_COUNT : 2
% GENERATED_ON : 2026-01-19 06:21
% CONTENT-ONLY FILE - NO PACKAGES, NO MACROS
% ==================================================

\BeginCourse{ECE101}
\CourseTitle{ECE101}{Sample Course}
\CourseMetaTable{ECE101 Sample Course}{{3}}{{1}}{{0}}{{0}}{4}{Course Code of Sample Pre-requisite Course Code of Sample Pre-requisite}{PM/TC}
\par \noindent \textbf{Course Description} \par \noindent
This course introduces fundamental circuit analysis methods for linear networks.
\par \noindent\textbf{Course Objectives}
\begin{itemize}
  \item Develop the ability to model simple electrical networks using ideal elements.
  \item Apply systematic methods to solve DC and AC linear circuits.
  \item Interpret circuit behaviour using standard engineering metrics.
  \item Develop the ability to model simple electrical networks using ideal elements.
  \item Apply systematic methods to solve DC and AC linear circuits.
  \item Interpret circuit behaviour using standard engineering metrics.
\end{itemize}
\par
\noindent
\textbf{Course Outcomes}
\begin{enumerate}[label=\textbf{CO\arabic*:}, leftmargin=*, nosep, topsep=0pt]
\item Analyse linear circuits using nodal and mesh methods.
\item Determine steady-state AC responses using phasor techniques.
\item Evaluate power in AC circuits using appropriate quantities.
\end{enumerate}
\par
\noindent\textbf{Articulation Matrix CO to PO, PSO}
\begin{longtblr}[
  entry = none, caption = {}, label = none
]{
  colspec = {| X[1.2,c,m] | *{11}{X[0.6, c,m] |} *{3}{X[0.6, c,m] |} },
  hlines = {0.5pt}, row{1,2} = {font=\bfseries}, rowhead = 2,
  abovesep = 0pt, belowsep = 0pt, rowsep = 1.5pt,  font = \small
}
\SetCell[r=2]{c} CO & \SetCell[c=11]{c} PO & & & & & & & & & &  & \SetCell[c=3]{c} PSO & \\
 & 1 & 2 & 3 & 4 & 5 & 6 & 7 & 8 & 9 & 10 & 11 & 1 & 2 & 3 \\
CO1 & 3 & 2 & 1 & 1 & - & - & - & - & - & - & - & 2 & 1 & 1 \\
CO2 & 2 & 3 & 2 & 1 & - & - & - & 2 & 1 & 1 & - & - & - & - \\
CO3 & 1 & 2 & 3 & 2 & 1 & - & - & - & - & - & 2 & 1 & 1 & - \\
\end{longtblr}
\par
\noindent\textbf{Articulation Matrix CO to SO, PSO}
\begin{longtblr}[ entry = none, caption = {}, label = none]{colspec = {| X[1.2,c,m] | *{7}{X[0.6,c,m] |} *{3}{X[0.6,c,m] |} },  hlines = {0.5pt},  row{1,2} = {font=\bfseries}, rowhead = 2, abovesep = 0pt, belowsep = 0pt, font = \small,}
\SetCell[r=2]{c} CO & \SetCell[c=7]{c} SO & & & & & & & \SetCell[c=3]{c} PSO & & \\
 & 1 & 2 & 3 & 4 & 5 & 6 & 7 & 1 & 2 & 3 \\
CO1 & 3 & 2 & 1 & 1 & - & - & - & 2 & 1 & 1 \\
CO2 & 2 & 3 & 2 & 1 & - & 2 & 1 & 1 & - & - \\
CO3 & 1 & 2 & 3 & 2 & 1 & - & - & 1 & 1 & - \\
\end{longtblr}
\par 
\noindent
\textbf{Textbooks}\begin{enumerate}
  \item Sedra, A. S., and Smith, K. C., \textit{Microelectronic Circuits}, 7th ed., Oxford University Press, 2016.
  \item Boylestad, R. L., and Nashelsky, L., \textit{Electronic Devices and Circuit Theory}, 11th ed., Pearson, 2013.
\end{enumerate}\textbf{References}\begin{enumerate}
  \item Millman, J., and Halkias, C., \textit{Integrated Electronics}, McGraw-Hill, 1972.
  \item IEEE Std 315-1975, Graphic Symbols for Electrical and Electronics Diagrams.
\end{enumerate}
\BeginCourse{ECE102}
\CourseTitle{ECE102}{Course Code of Sample New Courses}
\CourseMetaTable{ECE102 Course Code of Sample New Courses}{{3}}{{1}}{{0}}{{0}}{4}{Course Code of Sample Pre-requisite}{PE/TC}
\par \noindent \textbf{Course Description} \par \noindent
This course introduces fundamental circuit analysis methods for linear networks.
\par \noindent\textbf{Course Objectives}
\begin{itemize}
  \item Develop the ability to model simple electrical networks using ideal elements.
  \item Apply systematic methods to solve DC and AC linear circuits.
  \item Interpret circuit behaviour using standard engineering metrics.
\end{itemize}
\par
\noindent
\textbf{Course Outcomes}
\begin{enumerate}[label=\textbf{CO\arabic*:}, leftmargin=*, nosep, topsep=0pt]
\item Analyse linear circuits using nodal and mesh methods.
\item Determine steady-state AC responses using phasor techniques.
\item Evaluate power in AC circuits using appropriate quantities.
\end{enumerate}
\par
\noindent\textbf{Articulation Matrix CO to PO, PSO}
\begin{longtblr}[
  entry = none, caption = {}, label = none
]{
  colspec = {| X[1.2,c,m] | *{11}{X[0.6, c,m] |} *{3}{X[0.6, c,m] |} },
  hlines = {0.5pt}, row{1,2} = {font=\bfseries}, rowhead = 2,
  abovesep = 0pt, belowsep = 0pt, rowsep = 1.5pt,  font = \small
}
\SetCell[r=2]{c} CO & \SetCell[c=11]{c} PO & & & & & & & & & &  & \SetCell[c=3]{c} PSO & \\
 & 1 & 2 & 3 & 4 & 5 & 6 & 7 & 8 & 9 & 10 & 11 & 1 & 2 & 3 \\
CO1 & 3 & 2 & 1 & 1 & - & - & 2 & 1 & 1 & - & - & - & - & - \\
CO2 & 2 & 3 & 2 & 1 & - & - & - & - & 2 & 1 & 1 & - & - & - \\
CO3 & 1 & 2 & 3 & 2 & 1 & 2 & 1 & 1 & - & - & - & - & - & - \\
\end{longtblr}
\par
\noindent\textbf{Articulation Matrix CO to SO, PSO}
\begin{longtblr}[ entry = none, caption = {}, label = none]{colspec = {| X[1.2,c,m] | *{7}{X[0.6,c,m] |} *{3}{X[0.6,c,m] |} },  hlines = {0.5pt},  row{1,2} = {font=\bfseries}, rowhead = 2, abovesep = 0pt, belowsep = 0pt, font = \small,}
\SetCell[r=2]{c} CO & \SetCell[c=7]{c} SO & & & & & & & \SetCell[c=3]{c} PSO & & \\
 & 1 & 2 & 3 & 4 & 5 & 6 & 7 & 1 & 2 & 3 \\
CO1 & 3 & 2 & 1 & 1 & - & - & - & 2 & 1 & 1 \\
CO2 & 2 & 3 & 2 & 1 & - & 2 & 1 & 1 & - & - \\
CO3 & 1 & 2 & 3 & 2 & 1 & - & - & 1 & 1 & - \\
\end{longtblr}
\par 
\noindent
\textbf{Textbooks}\begin{enumerate}
  \item Sedra, A. S., and Smith, K. C., \textit{Microelectronic Circuits}, 7th ed., Oxford University Press, 2016.
  \item Boylestad, R. L., and Nashelsky, L., \textit{Electronic Devices and Circuit Theory}, 11th ed., Pearson, 2013.
\end{enumerate}\textbf{References}\begin{enumerate}
  \item Millman, J., and Halkias, C., \textit{Integrated Electronics}, McGraw-Hill, 1972.
  \item IEEE Std 315-1975, Graphic Symbols for Electrical and Electronics Diagrams.
\end{enumerate}